% https://guides.nyu.edu/LaTeX/sample-document

\documentclass{article}
\usepackage[utf8]{inputenc}
\usepackage[margin=0.7in]{geometry}
\usepackage{fancyhdr}
\usepackage{amssymb}
\usepackage{amsmath}
\usepackage{pgfplots}

\pagestyle{fancy}
\fancyhead[l]{David Spivack}
\fancyhead[c]{Math 400: Section 4.3}
\fancyhead[r]{\today}
\fancyfoot[c]{\thepage}
\renewcommand{\headrulewidth}{0.2pt} %Creates a horizontal line underneath the header
\setlength{\headheight}{15pt} %Sets enough space for the headers
\pgfplotsset{compat=1.18}

% The preamble ends with the command \begin{document}
\begin{document}

\begin{description} % corresponding to bold captions
  \setlength\itemsep{5em}

  \item\textbf{Answer questions about function properties with graphs.}

        \begin{description} % corresponding to problem numbers
          \setlength\itemsep{3em}

          \item 1. See pg. 305

                a. (1,3), (4, 6)

                b. (0, 1), (3, 4)

                c. (0, 2)

                d. (2, 4), (4, 6) \space Note: intervals are separate even though they are adjacent.

                e. [2, 3] \space Note: non-differentiable doesn't mean there will be an inflection point
          \item 5. See pg. 305
                a. increasing: (0, 1), (3, 5) decreasing: (1, 3), (5, 6)
                b. We evaluate roots and undefined inputs, not endpoints, for local extrema. If $f''(x) < 0$, extrema is a local max. The opposite holds true for local min.

                \fbox{max: $f(1)$ and $f(5)$,\space min: $f(3)$}
        \end{description}
  \item\textbf{Find intervals, extrema, concavity (requirements vary)}
        \begin{description}
          \setlength\itemsep{3em}
          \item 9. $2x^3 - 15x^2 + 24x - 5$

                Polynomials are differentiable for all real numbers.
                \begin{align*}
                  f'(x) & = 6x^2 - 30x + 24 &  & \text{find derivative} \\
                  0     & = 6x^2 - 30x + 24                             \\[2em]
                  0     & = x^2 -5x + 4     &  & \text{find root(s)}    \\
                  0     & = (x - 1)(x - 4)
                \end{align*}
                Critical values are 1 and 4. These are fence-posts. Test a number from each of the three ``fences''.
                \begin{align*}
                  f(0) & = (0 - 1)(0 - 4) &  & \text{interval (-$\infty$, 1)} \\
                  f(0) & = (-)(-)         &  & \text{increasing}              \\[2em]
                  f(2) & = (2 - 1)(2 - 4) &  & \text{interval (1, 4)}         \\
                  f(2) & = (+)(-)         &  & \text{decreasing}              \\[2em]
                  f(5) & = (5 - 1)(5 - 4) &  & \text{interval (4, $\infty$)}  \\
                  f(5) & = (+)(+)         &  & \text{increasing}
                \end{align*}
                \fbox{Therefore, function is increasing on ($-\infty$, 1)$\cup$(4, $\infty$) and decreasing on (1, 4).}\\

                Increasing to decreasing is a local max, while decreasing to increasing is a local min. We can take the second derivative to show this.
                \begin{align*}
                  f''(x) & = 12x - 30   &  & \text{find second derivative} \\[2em]
                  f''(1) & = 12(1) - 30 &  & \text{critical value \#1}     \\
                  f''(1) & = -18        &  & \text{max}                    \\[2em]
                  f''(4) & = 12(4) - 30 &  & \text{critical value \#2}     \\
                  f''(4) & = 18         &  & \text{min}
                \end{align*}
                Next, plug in both critical values to find values of local extrema at the critical values.
                \begin{align*}
                  f(1) & = 2(1)^3 - 15(1)^2 + 24(1) - 5 &  & \text{critical value \#1} \\
                  f(1) & = 2 - 15 + 24 - 5                                             \\
                  f(1) & = 6                                                           \\[2em]
                  f(4) & = 2(4)^3 - 15(4)^2 + 24(4) - 5 &  & \text{critical value \#2} \\
                  f(4) & = 128 - 240 + 96 - 5                                          \\
                  f(4) & = -21
                \end{align*}
                \fbox{max: $f(1) = 6$,\space min: $f(4) = - 21$}
          \item 13. $\dfrac{x^2-24}{x - 5}$
                \begin{align*}
                  f'(x) & = \frac{(x - 5)(2x) - (x^2-24)(1)}{(x - 5)^2} &  & \text{find derivative with quotient rule} \\
                  f'(x) & = \frac{2x^2 - 10x - x^2 + 24}{(x - 5)^2}                                                    \\
                  f'(x) & = \frac{x^2 - 10x + 24}{(x - 5)^2}                                                           \\
                \end{align*}
                Set factor in denominator to 0 and evaluate to find undefined inputs. Set numerator to 0 and
                evaluate to find roots. These will be the critical numbers.
                \begin{align*}
                  0 & = x^2 - 10x + 24   &  & \text{denominator} \\
                  x & = (x - 4)(x - 6)                           \\
                  x & = 4 \text{ and } 6                         \\[2em]
                  0 & = x - 5            &  & \text{numerator}   \\
                  x & = 5
                \end{align*}
                Critical numbers are 4, 5, 6. Use inputs that fall in the intervals we want to verify. We only need
                sign, so separate into factors if calculator is not an option.
                Plug into derivative.
                \begin{align*}
                  f'(0)   & = \frac{0^2 - 10(0) + 24}{(0 - 5)^2}           &  & \text{interval $(-\infty, 4)$} \\
                  f'(0)   & = \frac{+}{+}                                  &  & \text{increasing}              \\[2em]
                  f'(4.5) & = \frac{(4.5)^2 - 10(4.5) + 24}{((4.5) - 5)^2} &  & \text{interval $(4, 5)$}       \\
                  f'(4.5) & = \frac{-}{+}                                  &  & \text{decreasing}              \\[2em]
                  f'(5.5) & = \frac{(5.5)^2 - 10(5.5) + 24}{((5.5) - 5)^2} &  & \text{interval $(5, 6)$}       \\
                  f'(5.5) & = \frac{-}{+}                                  &  & \text{decreasing}              \\[2em]
                  f'(10)  & = \frac{10^2 - 10(10) + 24}{(10 - 5)^2}        &  & \text{interval $(6, \infty)$}  \\
                  f'(10)  & = \frac{+}{+}                                  &  & \text{increasing}
                \end{align*}
                Points where sign of derivative changes include a local max: $f(4)$ \space and a local min: $f(6)$
                \begin{align*}
                  f(4) & = \frac{(4)^2-24}{4 - 5} \\
                  f(4) & = \frac{-4}{-1}          \\
                  f(4) & = 4                      \\[2 em]
                  f(6) & = \frac{(6)^2-24}{6 - 5} \\
                  f(6) & = \frac{12}{1}           \\
                  f(6) & = 12
                \end{align*}
                \fbox{local max: $f(4) = 8$, local min: $f(6) = 12$}
          \item 17. $x^3-3x^2 - 9x + 4$

                Polynomials are differentiable for all real numbers.
                \begin{align*}
                  f'(x) & = 3x^2-6x - 9     &  & \text{find derivative} \\[2em]
                  0     & = 3x^2-6x - 9     &  & \text{find root(s)}    \\
                  0     & = x^2-2x - 3                                  \\
                  0     & = (x + 1) (x - 3)
                \end{align*}
                Critical numbers are -1 and 3
                \begin{align*}
                  f'(-2) & = 3(-2)^2-6(-2) - 9 &  & \text{interval $(-\infty, -1)$} \\
                  f'(-2) & = 15                &  & \text{increasing}               \\[2em]
                  f'(0)  & = 3(0)^2-6(0) - 9   &  & \text{interval $(-1, 3)$}       \\
                  f'(0)  & = -9                &  & \text{decreasing}               \\[2em]
                  f'(4)  & = 3(4)^2-6(4) - 9   &  & \text{interval $(3, \infty)$}   \\
                  f'(4)  & = 24                &  & \text{increasing}
                \end{align*}

                Local max: $f(-1)$, Local min: $f(3)$. Plug back into original function. We can
                skip this step if just trying to find concavity and inflection point.
                \begin{align*}
                  f(-1) & = (-1)^3-3(-1)^2 - 9(-1) + 4 \\
                  f(-1) & = -1 - 3 + 9 + 4             \\
                  f(-1) & = 9                          \\[2em]
                  f(3)  & = (3)^3-3(3)^2 - 9(3) + 4    \\
                  f(3)  & = 27 - 27 - 27 + 4           \\
                  f(3)  & = -23
                \end{align*}
                max: $f(-1) = 9$,\space min: $f(3) = -23$

                Next, we will work with second derivative.
                \begin{align*}
                  f'(x)  & = 3x^2-6x - 9                                    \\
                  f''(x) & = 6x - 6      &  & \text{find second derivative} \\[2em]
                  0      & = 6x - 6      &  & \text{find roots}             \\
                  x      & = 1
                \end{align*}

                Fence posts are the roots. Evaluate inputs in the ``fences''. Negative means concave
                down and positive means concave up.
                \begin{align*}
                  f''(0) & = 6(0) - 6  &  & \text{interval $(-\infty, 1)$} \\
                  f''(0) & = -6        &  & \text{concave down}            \\[2em]
                  f''(2) & =  6(2) - 6 &  & \text{interval $(-1, \infty)$} \\
                  f''(2) & = 12        &  & \text{concave up}
                \end{align*}

                Find inflection point by plugging in root of second derivative into original equation.
                \begin{align*}
                  f(1) & = (1)^3-3(1)^2 - 9(1) + 4 \\
                  f(1) & = 1 - 3 - 9 + 4           \\
                  f(1) & = -7
                \end{align*}
                \fbox{Inflection point: (1, -7)}
          \item 21. $\ln{(x^2+5)}$

                Function has natural log so domain is all real numbers greater than 0.
                \begin{align*}
                  f'(x)  & =  \frac{1}{x^2+5}(2x)                     &  & \text{find derivative}        \\
                  f'(x)  & =  \frac{2x}{x^2+5}                        &  &                               \\[2em]
                  f''(x) & =  \frac{(x^2+5)(2) - (2x)(2x)}{(x^2+5)^2} &  & \text{find second derivative} \\
                  f''(x) & =  \frac{2x^2+10 - 4x^2}{(x^2+5)^2}                                           \\
                  f''(x) & =  \frac{-2x^2 + 10}{(x^2+5)^2}
                \end{align*}
                Denominator can never be 0. Set numerator to 0 to find roots, aka the $x$-values of inflection points.
                \begin{align*}
                  0 = -2x^2 + 10 \\
                  2x^2 = 10      \\
                  x^2 = 5        \\
                  x = \pm \sqrt{5}
                \end{align*}
                Evaluate  the ``fences''.
                \begin{align*}
                  f''(-3) & =  \frac{-2(-3)^2 + 10}{((-3)^2+5)^2} &  & \text{interval $(-\infty, -\sqrt{5})$}  \\
                  f''(-3) & = \frac{-}{+}                         &  & \text{concave down}                     \\[2em]
                  f''(0)  & =  \frac{-2(0)^2 + 10}{(0^2+5)^2}     &  & \text{interval $(-\sqrt{5}, \sqrt{5})$} \\
                  f''(0)  & = \frac{+}{+}                         &  & \text{concave up}                       \\[2em]
                  f''(3)  & =  \frac{-2(3)^2 + 10}{(3^2+5)^2}     &  & \text{interval $(\sqrt{5}, \infty)$}    \\
                  f''(3)  & = \frac{-}{+}                         &  & \text{concave down}
                \end{align*}
                Plug solutions back into original equation to find inflection points.
                \begin{align*}
                  f(-\sqrt{5}) & = \ln{((-\sqrt{5})^2+5)} \\
                  f(-\sqrt{5}) & = \ln{(5 + 5)}           \\
                  f(-\sqrt{5}) & = \ln{10}                \\[2em]
                  f(\sqrt{5})  & =\ln{(\sqrt{5}^2+5)}     \\
                  f(\sqrt{5})  & = \ln{(5 + 5)}           \\
                  f(\sqrt{5})  & =\ln{10}
                \end{align*}
                \fbox{Inflection points: $(\pm\sqrt{5}, \ln{10})$}
          \item 25. $x^2 - x - \ln{x}$

                Function has natural log so domain is all real numbers greater than 0.
                \begin{align*}
                  f'(x) & = 2x - 1 - \frac{1}{x}                             &  & \text{find derivative}                  \\[2em]
                  0     & = 2x - 1 - \frac{1}{x}                             &  & \text{find root(s) of first derivative} \\
                  0     & = 2x * \frac{x}{x} - 1 * \frac{x}{x} - \frac{1}{x} &  & \text{get common denominators}          \\
                  0     & = \frac{2x^2 - x - 1}{x}                                                                        \\[2em]
                  0     & = 2x^2 - x - 1                                     &  & \text{set numerator = 0}                \\
                  0     & = 2x(x - 1) + (1)(x - 1)                                                                        \\
                  0     & = (2x + 1)(x - 1)                                  &  & \text{factors}
                \end{align*}

                By evaluating roots and inputs that make derivative undefined, the only critical value is shown to be 1. $-\frac{1}{2}$ and 0 are not
                included because they aren't in domain of original function. Next, evaluate inputs on intervals surrounding the critical value.
                \begin{align*}
                  f'(\frac{1}{2}) & = 2\left(\frac{1}{2}\right) - 1 - \frac{1}{\frac{1}{2}} &  & \text{interval $(0, 1)$}      \\
                  f'(\frac{1}{2}) & = 1 - 1 - 2                                                                                \\
                  f'(\frac{1}{2}) & = -2                                                    &  & \text{decreasing}             \\[2em]
                  f'(2)           & = 2(2) - 1 - \frac{1}{2}                                &  & \text{interval $(1, \infty)$} \\
                  f'(2)           & = 4 - 1 -  \frac{1}{2}                                                                     \\
                  f'(2)           & = -2.5                                                  &  & \text{increasing}
                \end{align*}

                Plug in critical values to original function.

                \begin{align*}
                  f(0) & = 0^2 - 0 - \ln{0} \\
                \end{align*}
                \fbox{No local max. local min: $f(1) = 0$}
                \begin{align*}
                  f''(x) & = 2 +x^{-2}           &  & \text{find second derivative} \\
                  f''(x) & = 2 + \frac{1}{x^{2}}
                \end{align*}
                Solution is always positive. There are no real roots, so no real inflection points. Therefore the function is concave upwards
                everywhere on its domain.
        \end{description}
  \item\textbf{Find local max and min with first and second derivative test}
        \begin{description}
          \setlength\itemsep{3em}
          \item 29. $f(x) = 1 + 3x^2 - 2x^3$

                Function is a polynomial so it is differentiable for all real numbers.
                \begin{align*}
                  f'(x) & = -6x^2 + 6x &  & \text{find derivative} \\[2em]
                  0     & = -6x^2 + 6x &  & \text{find root(s)}    \\
                  0     & = -6x(x - 1)
                \end{align*}
                Critical numbers are 0 and 1.

                First derivative test:
                \begin{align*}
                  f'(-1)          & = -6(-1)^2 + 6(-1)                                         &  & \text{interval $(-\infty, 0)$} \\
                  f'(-1)          & = -6 - 6                                                   &  & \text{decreasing}              \\[2em]
                  f'(\frac{1}{2}) & = -6\left(\frac{1}{2}\right)^2 + 6\left(\frac{1}{2}\right) &  & \text{interval $(0, 1)$}       \\
                  f'(\frac{1}{2}) & = -2\frac{1}{2}  +   3                                     &  & \text{increasing}              \\[2em]
                  f'(10)          & = -6(10)^2 + 6(10)                                         &  & \text{interval $(1, \infty)$}  \\
                  f'(10)          & = -600 + 60                                                &  & \text{decreasing}
                \end{align*}
                local min at $f(0)$, local max at $f(1)$.

                Second derivative test:
                \begin{align*}
                  f''(x) & = -12x + 6   &  & \text{find derivative}                           \\[2em]
                  f''(0) & = -12(0) + 6 &  & \text{critical value \#1}                        \\
                  f''(0) & = 6          &  & \text{positive so this is occurs at a local min} \\[2em]
                  f''(1) & = -12(1) + 6 &  & \text{critical value \#2}                        \\
                  f''(1) & = -12 + 6    &  & \text{negative so this occurs at a local max}
                \end{align*}

                We've found the local min and max locations. Now plug into original function to get extrema values.
                \begin{align*}
                  f(0) & = 1 + 3(0)^2 - 2(0)^3 \\
                  f(0) & = 1                   \\[2em]
                  f(1) & = 1 + 3(1)^2 - 2(1)^3 \\
                  f(1) & = 1 + 3 - 2           \\
                  f(1) & = 2
                \end{align*}
                \fbox{local max: $f(1) = 2$,  local min: $f(0) = 1$}
        \end{description}
\end{description}
\end{document}
