% https://guides.nyu.edu/LaTeX/sample-document

\documentclass{article}
\usepackage[utf8]{inputenc}
\usepackage[margin=0.7in]{geometry}
\usepackage{fancyhdr}
\usepackage{amssymb}
\usepackage{amsmath}
\usepackage{pgfplots}

\pagestyle{fancy}
\fancyhead[l]{David Spivack}
\fancyhead[c]{Math 400: Section 4.1}
\fancyhead[r]{\today}
\fancyfoot[c]{\thepage}
\renewcommand{\headrulewidth}{0.2pt} %Creates a horizontal line underneath the header
\setlength{\headheight}{15pt} %Sets enough space for the headers
\pgfplotsset{compat=1.18}

% The preamble ends with the command \begin{document}
\begin{document}

\begin{description} % corresponding to bold captions
    \setlength\itemsep{5em}

    \item\textbf{Find extrema on displayed graph.}

          \begin{description} % corresponding to problem numbers
              \setlength\itemsep{3em}

              \item 3. referenced graph pg. 286
                    \begin{description}
                        \item min: r
                        \item maxmax: s
                        \item loc min: r, b
                        \item loc max: c
                        \item neither: d, a
                    \end{description}

              \item 5. referenced graph pg. 286
                    \begin{description}
                        \item min: undefined
                        \item maxmax: 5
                        \item loc min: 2, 3
                        \item loc max: 5
                    \end{description}
          \end{description}

    \item\textbf{Sketch a graph with the given constraints.}

          7. Absolute maximum at 5, absolute minimum at 2, local maximum at 3, local minima at 2 and 4. \\

          \begin{tikzpicture}
              \begin{axis}
                  \addplot+ [
                      smooth,
                  ] coordinates {
                          (1,3) (2,1) (3,3.5) (4,2.5) (5,4)
                      };
              \end{axis}
          \end{tikzpicture}

    \item\textbf{Create graph from function and find extrema.}

          7. $f(x) = \begin{cases}
                  x^2    & \text{if } -1 \leq x \leq 0, \\
                  2 - 3x & \text{if } 0 < x \leq 1
              \end{cases}$

          \begin{tikzpicture}
              \begin{axis}
                  \addplot[domain=-1:0] {x^2};
                  \addplot[domain=0:1] {2 - 3*x};
                  \addplot[mark=*] coordinates {(-1,1)};
                  \addplot[mark=*] coordinates {(0, 0)};
                  \addplot[mark=*] coordinates {(1,-1)};
                  \addplot[mark=*,fill=white] coordinates {(0,2)};
              \end{axis}
          \end{tikzpicture} \\

          Absolute min = -1, Local min = 0. No abs or local max.
    \item\textbf{Find the critical numbers of the function.}
          \begin{description}
              \setlength\itemsep{3em}
              \item 31. $3x^4 + 8x^3-48x^2$
                    Polynomials are differentiable for all  $x \in \mathbb{R}$.
                    \begin{align*}
                        f'(x) & = 4x^3 + 24x^2-96x &  & \text{find derivative} \\
                        0     & = x^3 + 2x^2 - 8x  &  & \text{find root(s)}    \\
                        0     & = x(x^2 +2x - 8)                               \\
                        0     & = x(x + 4)(x-2)
                    \end{align*}
                    \fbox{$x = $ 0, 4, and 2}
              \item 35. $g(y) = \frac{y-1}{y^2 - y + 1}$
                    \begin{align*}
                        g'(y) & = \frac{(y^2 - y + 1)(1) - (y-1)(2y-1)}{{(y^2 - y + 1)}^2} &  & \text{quotient rule} \\ g'(y) &= && \text{how to simplify from here?}
                    \end{align*}
              \item 39. $h(t) = t^{3/4} - 2t^{1/4}$
                    \begin{align*}
                        h'(t) & = \frac{3}{4}t^{-1/4} - \frac{1}{2}t^{-3/4}                                               &  & \text{find derivative}         \\[1em]
                        h'(t) & = \frac{3}{4t^{1/4}} - \frac{1}{2t^{3/4}}                                                 &  & \text{rewrite exponents}       \\[1em]
                        h'(t) & = \frac{3}{4t^{1/4}} \cdot \frac{t^{2/4}}{t^{2/4}} - \frac{1}{2t^{3/4}} \cdot \frac{2}{2} &  & \text{get common denominators} \\[1em]
                        h'(t) & = \frac{3t^{1/2} - 2}{4t^{3/4}}
                    \end{align*}
                    \begin{align*}
                        0 & = 4t^{3/4} &  & \text{set denominator to 0 to evaluate where function is undefined} \\
                        t & = 0
                    \end{align*}
                    \begin{align*}
                        0       & = 3t^{1/2} - 2                          &  & \text{set numerator to 0 to find root(s)} \\[1em]
                        t^{1/2} & = \frac{2}{3} \implies  t = \frac{4}{9}
                    \end{align*}
                    \fbox{$t = 0$ and $\frac{4}{9}$}

              \item 43. $f(x) = x^{1/3}(4-x)^{2/3}$
                    \begin{align*}
                        f'(x) & = x^{1/3}(-4 + x)(-1) + (4-x)^{2/3}\frac{1}{3}x^{2/3}                                    &  & \text{product rule}                   \\[1em]
                        f'(x) & = \frac{x^{1/3}(-4 + x)}{3} + \frac{(4-x)^{2/3}}{3x^{2/3}}                               &  & \text{rewrite as rational expression} \\[1em]
                        f'(x) & = \frac{x^{1/3}(-4 + x)}{3} \cdot \frac{x^{2/3}}{x^{2/3}} + \frac{(4-x)^{2/3}}{3x^{2/3}} &  & \text{get common denominators}        \\[1em]
                        f'(x) & = \frac{x(-4 + x) + (4-x)^{2/3}}{3x^{2/3}}
                    \end{align*}
                    \begin{align*}
                        0 & = 3x^{2/3} &  & \text{set denominator to 0 to evaluate where function is undefined} \\
                        x & = 0
                    \end{align*}
                    \begin{align*}
                        0 & = x(-4 + x) + (4-x)^{2/3}     &  & \text{set numerator to 0 to find root(s)}                 \\
                        0 & = (4 - x)(-x + (4 - x)^{1/3}) &  & \text{find factors, $x=4$}                                \\
                        0 & = -x + (4 - x)^{1/3}          &  & \text{isolate factor by removing $(4-x)$}                 \\
                        x & =                             &  & \text{how to simplify from here? missing $x=\frac{4}{3}$}
                    \end{align*}
                    \fbox{$x = 0$ and 4}
              \item 47. $g(x) = x^2\ln{x}$
                    \begin{align*}
                        g'(x)         & = x^2\left(\frac{1}{x}\right) + 2x\ln{x} &  & \text{product rule}                            \\
                        g'(x)         & = x + 2x\ln{x}                                                                               \\
                        0             & = x + 2x\ln{x}                           &  & \text{find root(s)}                            \\
                        -x            & = 2x\ln{x}                                                                                   \\
                        \frac{-x}{2x} & = ln{x}                                  &  & \text{$x$'s on left side of expression cancel} \\
                        -\frac{1}{2}  & = ln{x}                                                                                      \\
                        e^{-1/2}      & = e^{ln{x}} = x                          &  & \text{use e to isolate remaining x}            \\
                    \end{align*}
                    \fbox{$x = \frac{1}{\sqrt{e}}$}

          \end{description}
    \item\textbf{Find the value of $f(x)$ at absolute min and max of the function on the given interval.}
          \begin{description}
              \setlength\itemsep{3em}
              \item 51. $f(x) = 12 + 4x - x^2$, \space [0, 5]
                    \begin{align*}
                        f'(x) & = -2x + 4 &  & \text{find derivative} \\
                        0     & = -2x + 4 &  & \text{find root(s)}    \\
                        -4    & = -2x     &  & \text{root is 2}       \\
                    \end{align*}
                    Original function is a polynomial so it is continuous for all real numbers.
                    Plug in roots and endpoints to original function.
                    \begin{align*}
                        f(2) & = 12 + 4(2) - (2)^2 = 12 + 8 - 4 =  16 \\
                        f(0) & = 12 + 4(0) - (0)^2 = 12 + 0 - 0 = 12  \\
                        f(5) & = 12 + 4(5) - (5)^2 = 12 + 20 - 25 = 7
                    \end{align*}
                    \fbox{max: $f(2) = 16$, \space min: $f(5) = 7$}
              \item 55. $f(x) = 3x^4 - 4x^3 - 12x^2 + 1$, \space [-2, 3]
                    \begin{align*}
                        f'(x) & = 12x^3 - 12x^2 - 24x &  & \text{find derivative}        \\
                        0     & = 12x^3 - 12x^2 - 24x &  & \text{find roots}             \\
                        0     & = (x^2 - x - 2)(12x)  &  & \text{find factors}           \\
                        0     & = (x - 2)(x + 1)(12x) &  & \text{roots are 2, -1, and 0}
                    \end{align*}
                    Original function is polynomial so it is continuous for all real numbers. Plug in roots and endpoints into original function.
                    \begin{align*}
                        f(-2) & = 3(-2)^4 - 4(-2)^3 - 12(-2)^2 + 1 &  & \text{critical val \#1} \\
                        f(-2) & = (3*16) - (4*-8) - (12*4) + 1                                  \\
                        f(-2) & = 48 - (-32) - 48 + 1                                           \\
                        f(-2) & = 33                                                            \\[2em]
                        f(3)  & = 3(3)^4 - 4(3)^3 - 12(3)^2 + 1    &  & \text{critical val \#2} \\
                        f(3)  & = (3*81) - (4*27) - (12*9) + 1                                  \\
                        f(3)  & = 243 - 108 - 108 + 1                                           \\
                        f(3)  & = 28                                                            \\[2em]
                        f(2)  & = 3(2)^4 - 4(2)^3 - 12(2)^2 + 1    &  & \text{critical val \#3} \\
                        f(2)  & = (3*16) - (4*8) - (12*4) + 1                                   \\
                        f(2)  & = 48 - 32 - 48 + 1                                              \\
                        f(2)  & = -31                                                           \\[2em]
                        f(0)  & = 3(0)^4 - 4(0)^3 - 12(0)^2 + 1    &  & \text{critical val \#4} \\
                        f(0)  & = 1                                                             \\[2em]
                        f(-1) & = 3(-1)^4 - 4(-1)^3 - 12-(1)^2 + 1 &  & \text{critical val \#5} \\
                        f(-1) & = 3 - 4 + 12 + 1                                                \\
                        f(-1) & = 12
                    \end{align*}
                    \fbox{max: $f(-2) = 33$, \space min: $f(2) = -31$}
              \item 59. $f(x) = t - \sqrt[3]{t}$, \space [-1, 4]
                    \begin{align*}
                        f(x)  & = t - t^{1/3}             &  & \text{rewrite function with exponents} \\[1em]
                        f'(x) & = 1 - \frac{1}{3}t^{-2/3} &  & \text{find derivative}                 \\[1em]
                        f'(x) & = 1 - \frac{1}{3t^{2/3}}  &  & \text{simplify exponents}
                    \end{align*}
                    Radical is undefined when denominator is 0, so $f(0)$ is a critical value. Set denominator to 1 to find the root, where the function evaluates to 0.
                    \[3t^{2/3} = 1 \quad \implies \quad t^{2/3} = \frac{1}{3}\quad \implies \quad t = \frac{\sqrt{1^3}}{\sqrt{3^3}} = \frac{\sqrt{3}}{9}\]
                    Plug in critical values  into original equation.
                    \begin{align*}
                        f(-1)                            & = -1 - \sqrt[3]{-1}                                      &  & \text{critical val \#1} \\
                        f(-1)                            & = -1 + 1                                                                              \\
                        f(-1)                            & = 0                                                                                   \\[2em]
                        f(4)                             & = 4 - \sqrt[3]{4}                                        &  & \text{critical val \#2} \\
                        f(4)                             & = 4 - 1.58 = 2.41                                                                     \\[2em]
                        f(0)                             & = 0 - \sqrt[3]{0} = 0                                    &  & \text{critical val \#3} \\[2em]
                        f\left(\frac{\sqrt{3}}{9}\right) & = \frac{\sqrt{3}}{9} - \sqrt[3]{\frac{\sqrt{3}}{9}}      &  & \text{critical val \#4} \\[1em]
                        f\left(\frac{\sqrt{3}}{9}\right) & = \frac{\sqrt{3}}{9} - \left(3^{-2}*3^{1/2}\right)^{1/3}                              \\[1em]
                        f\left(\frac{\sqrt{3}}{9}\right) & = \frac{\sqrt{3}}{9} - \left(3^{-2/3} * 3 ^ {1/6}\right)                              \\[1em]
                        f\left(\frac{\sqrt{3}}{9}\right) & = \frac{\sqrt{3}}{9} - 3^{-3/6}                                                       \\[1em]
                        f\left(\frac{\sqrt{3}}{9}\right) & = \frac{\sqrt{3}}{9} - \frac{1}{\sqrt{3}}                                             \\[1em]
                        f\left(\frac{\sqrt{3}}{9}\right) & = -0.38
                    \end{align*}
                    \fbox{max: $f(4) = 2.41$, \space min: $f\left(\frac{\sqrt{3}}{9}\right) = -0.38$}
              \item 63. $f(x) = x^{-2}\ln{x}$, \space [$\frac{1}{2}$, 4]
                    \begin{align*}
                        f'(x)       & = x^{-2}\left(\frac{1}{x}\right) + -2x^{-3}\ln{x} &  & \text{find derivative}               \\
                        f'(x)       & = x^{-3} - 2x^{-3}\ln{x}                                                                    \\
                        f'(x)       & = x^{-3} (1 - 2\ln{x})                                                                      \\
                        0           & = x^{-3} (1 - 2\ln{x})                            &  & \text{find root(s)}                  \\
                        0           & =  1 - 2\ln{x}                                    &  & \text{remove factor, \space $x = 0$} \\
                        \frac{1}{2} & = ln{x}                                                                                     \\
                        e^{1/2}     & = x
                    \end{align*}
                    Critical numbers are $e^{1/2}$, $\frac{1}{2}$, and 4. 0 is not included because it is not in domain of original function. Plug critical numbers into original function and evaluate.
                    \begin{align*}
                        f(e^{1/2})                & = (e^{1/2})^{-2}\ln{e^{1/2}}                    &  & \text{critical val \#2} \\
                        f(e^{1/2})                & = (e^{-1})\left(\frac{1}{2}\right)                                           \\
                        f(e^{1/2})                & = \left(\frac{1}{2e}\right)                                                  \\
                        f(e^{1/2})                & = 0.18                                                                       \\[2em]
                        f\left(\frac{1}{2}\right) & = \left(\frac{1}{2}\right)^{-2}\ln{\frac{1}{2}} &  & \text{critical val \#3} \\
                        f\left(\frac{1}{2}\right) & = 4\ln{\frac{1}{2}}                                                          \\
                        f\left(\frac{1}{2}\right) & = -2.77                                                                      \\[2em]
                        f(4)                      & = 4^{-2}\ln{4}                                  &  & \text{critical val \#4} \\
                        f(4)                      & = \frac{\ln{4}}{16}                                                          \\
                        f(4)                      & = 0.09
                    \end{align*}
                    \fbox{max: $f(e^{1/2}) = 0.18$, \space min: $f\left(\frac{1}{2}\right) = -5.77$}
          \end{description}
    \item\textbf{(a) Use a graph to estimate extrema to 2 decimal places (b) Find extrema using calculus. }
          \begin{description}
              \setlength\itemsep{3em}
              \item 70. $f(x) = e^x + e^{-2x}$, \space $0 \leq x \leq 1$

                    \begin{tikzpicture}
                        \begin{axis}
                            \addplot[domain=0:1] {e^x + e^(-2*x)};
                        \end{axis}
                    \end{tikzpicture} \\

                    (a) \fbox{max: $f(1) = 2.85$, \space min: $f(0.23) = 0.89$}

                    \begin{align*}
                        f'(x) & = e^x + e^{-2x}(-2) &  & \text{find derivative}                                        \\
                        f'(x) & =                   &  & \text{How to simplify without taking log of negative number?} \\
                    \end{align*}
          \end{description}
\end{description}
\end{document}
