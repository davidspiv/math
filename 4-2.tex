% https://guides.nyu.edu/LaTeX/sample-document

\documentclass{article}
\usepackage[utf8]{inputenc}
\usepackage[margin=0.7in]{geometry}
\usepackage{fancyhdr}
\usepackage{amssymb}
\usepackage{amsmath}
\usepackage{pgfplots}

\pagestyle{fancy}
\fancyhead[l]{David Spivack}
\fancyhead[c]{Math 400: Section 4.2}
\fancyhead[r]{\today}
\fancyfoot[c]{\thepage}
\renewcommand{\headrulewidth}{0.2pt} %Creates a horizontal line underneath the header
\setlength{\headheight}{15pt} %Sets enough space for the headers
\pgfplotsset{compat=1.18}

% The preamble ends with the command \begin{document}
\begin{document}

\begin{description} % corresponding to bold captions
      \setlength\itemsep{5em}

      \item\textbf{Find the value $c$ that satisfies the Mean Value Theorem.}

            \text{Mean value theorem: } \[f'(c) = \frac{f(b) - f(a)}{b - a}\]
            \begin{description} % corresponding to problem numbers
                  \setlength\itemsep{3em}

                  \item 7. See graph pg. 295
                        \begin{align*}
                              f'(c) & = \frac{f(0) - f(5)}{0 - 5} \\
                              f'(c) & = \frac{1 - 3}{0 - 5}       \\
                              f'(c) & = \frac{2}{5}
                        \end{align*}
                        We know the graph has a slope of $\frac{2}{5}$ at c, therefore $3<c<4$. How does the book narrow to 3.8?
                  \item 8. See graph pg. 295. Mean value theorem does not apply. Function is non-differentiable at 4.
            \end{description}

      \item\textbf{Verify the function satisfies Rolle's Theorem and find all numbers $c$.}
            \begin{description}
                  \item 11. $f(x) = \sin{(x/2)}$, \space [$\pi/2$, \space $3\pi/2$]

                        Function is continuous on the open interval, differentiable on the closed interval, and $f(\pi/2)=f(3\pi/2)$.

                        c = $\pi$ because $f(\pi) = 0$ and $\pi$ is in the interval.
            \end{description}
      \item\textbf{Find c with Mean Value Theorem.} % empty item to keep spacing on problems with no
            \begin{description}
                  \item   21. $f(x) = (x-3)^{-2}$, \space [1, \space 4]
            \end{description}

            Function is not continuous at 3, therefore the MVT does not apply on this interval.
      \item\textbf{Verify the function has only one real solution.}
            \begin{description}
                  \item 23. $2x + \cos{x} = 0$

                        Function is differentiable for all real numbers. We apply Intermediate Value Theorem to
                        show there is at least one solution.
                        \begin{align*}
                              2\left(-\frac{\pi}{2}\right) + \cos{\left(-\frac{\pi}{2}\right)} = -\frac{2\pi}{2} - 1 &  & 2\left(\frac{\pi}{2}\right) + \cos{\left(\frac{\pi}{2}\right)} = \frac{2\pi}{2} + 1
                        \end{align*}
                        We get one positive and one negative number. Since $f(x)$ is continuous, we know that there must
                        be at least one root. Then, we show that the derivative has NO real roots. Hence, the graph is
                        always decreasing or increasing and never "turns" back towards the $x$ axis.
                        \begin{align*}
                              2 + -\sin{x} & = 0 &  & \text{implicit differentiation}
                        \end{align*}
                        $-1 \leq sin{x} \leq 1$, so there are no real solutions to the above equation.

                        Therefore, there can only be one solution to the original function.
                  \item 25. $x^3 - 15x + c = 0$ on the interval [-2, 2 ]
                        \begin{align*}
                              (-2)^3 - 15(-2) + c = 22 + c \\
                              (2)^3 - 15(2) + c = -22 + c
                        \end{align*}
                        $-2 \leq c \leq 2$, so regardless of what $c$ is, one solution is positive and one is negative.
                        \begin{align*}
                              3x^2 - 15 = 0 &  & \text{implicit differentiation} \\
                              3x^2 = 15                                          \\
                              x^2 = 5                                            \\
                              x = \pm \sqrt{5}
                        \end{align*}
                        The solutions to the derivative falls outside the interval. Thus, since the graph "turns" back
                        towards the x axis after the interval, the interval can not have more than one real solution.
            \end{description}
      \item\textbf{Show that a polynomial of degree $n$ has at most $n$ real roots.}
            \begin{description}
                  \item 27. Proof by induction. First, we define a one degree polynomial. Assuming $a \neq 0$
                        \begin{align*}
                              0 & = ax + b      \\
                              x & =-\frac{b}{a}
                        \end{align*}
                        Because $a$ can't be 0, this fraction always evaluates to a real number. Therefore this polynomial
                        has one real root. Next, we make this polynomial a degree higher.
                        \begin{align*}
                              f(x) & = (ax + b)(x - c)
                        \end{align*}
                        We know that $(x - c)$ has no more than 1 real solution, $c$. This second degree polynomial can't have more than two solutions
                        (one from the first degree polynomial, and at most one from our factor). We can continue this process to an $n$ degree polynomial.
                        Therefore, an $n$ degree polynomial has at most $n$ real solution(s).
            \end{description}
      \item\textbf{Prove.}
            \begin{description}
                  \item 33. $\sin{x} < x$ if $0 \leq x \leq 2\pi$

                        Both functions are differentiable on the open interval. We can use the MVT to compare mean instantaneous rates
                        of change.
                        \begin{align*}
                              f'(c) = \frac{f(2\pi) - f(0)}{2\pi - 0} &  & \text{MVT for } & f(x) = sin{x} \\
                              f'(c) = \frac{1 - 0}{2\pi - 0}                                               \\
                              f'(c) = \frac{1}{2\pi}                                                       \\[2em]
                              f'(c) = 1                               &  & \text{MVT for } & f(x) = x
                        \end{align*}
                        This tells us the mean slope for $sin(x)$ is less or equal to $x$ on the interval.

                        Next, find derivatives.
                        \begin{align*}
                              \cos{x} \leq 1
                        \end{align*}

                        $f'(x) = \cos{x}$ has only one real root, $f'(x) = 1$ has no real roots. We know that the slope
                        for both functions was the same initially (plug in start of interval to both derivatives), the slope of $x$ NEVER decreases, and the slope of
                        $\sin(x)$ ONLY decreases on this interval. Therefore $\sin{x} < x$ on the interval.
            \end{description}
\end{description}
\end{document}
