% https://guides.nyu.edu/LaTeX/sample-document

\documentclass{article}
\usepackage[utf8]{inputenc}
\usepackage[margin=0.7in]{geometry}
\usepackage{fancyhdr}
\usepackage{amssymb}
\usepackage{amsmath}
\usepackage{pgfplots}

\pagestyle{fancy}
\fancyhead[l]{David Spivack}
\fancyhead[c]{Math 400: Section 4.4}
\fancyhead[r]{\today}
\fancyfoot[c]{\thepage}
\renewcommand{\headrulewidth}{0.2pt} %Creates a horizontal line underneath the header
\setlength{\headheight}{15pt} %Sets enough space for the headers
\pgfplotsset{compat=1.18}

\DeclareMathOperator{\sech}{sech}
\DeclareMathOperator{\csch}{csch}
\DeclareMathOperator{\arcsec}{arcsec}
\DeclareMathOperator{\arccot}{arcCot}
\DeclareMathOperator{\arccsc}{arcCsc}
\DeclareMathOperator{\arccosh}{arcCosh}
\DeclareMathOperator{\arcsinh}{arcsinh}
\DeclareMathOperator{\arctanh}{arctanh}
\DeclareMathOperator{\arcsech}{arcsech}
\DeclareMathOperator{\arccsch}{arcCsch}
\DeclareMathOperator{\arccoth}{arcCoth}

% The preamble ends with the command \begin{document}
\begin{document}

\begin{description} % corresponding to bold captions
  \setlength\itemsep{5em}

  \item\textbf{State indeterminate form or evaluate limit.}

        Indeterminate forms:
        \begin{align*}
          \frac{0}{0} &  & \frac{\infty}{\infty} &  & 0 \cdot \infty &  & \infty - \infty &  & 0^0 &  & 1^{\infty} &  & \infty^0
        \end{align*}
        \begin{description}
          \setlength\itemsep{2em}
          \item 1.
                \begin{align*}
                  \text{a. } & \lim_{x\to a} \frac{f(x)}{g(x)} = \frac{0}{0}           &  & \text{b. } \lim_{x\to a} \frac{f(x)}{p(x)} = \frac{0}{\infty} =  0                         \\
                  \text{c. } & \lim_{x\to a} \frac{h(x)}{p(x)} = \frac{1}{\infty} = 0  &  & \text{d. } \lim_{x\to a} \frac{p(x)}{f(x)} = \frac{\infty}{0} = \text{false indeterminate} \\
                  \text{e. } & \lim_{x\to a} \frac{p(x)}{q(x)} = \frac{\infty}{\infty}
                \end{align*}
          \item 2.
                \begin{align*}
                  \text{a. } & \lim_{x\to a} [f(x)p(x)] = 0 \cdot \infty                                             \\
                  \text{b. } & \lim_{x\to a} [h(x)p(x)]= 1 \cdot \infty = 1\text{ ?}                                 \\
                  \text{c. } & \lim_{x\to a} [p(x)q(x)] = \infty \cdot \infty        &  & \text{false indeterminate}
                \end{align*}
          \item 3.
                \begin{align*}
                  \text{a. } & \lim_{x\to a} [f(x) - p(x)] = 0 + \infty = \infty                                 \\
                  \text{b. } & \lim_{x\to a} [p(x) - q(x)]= \infty - \infty      &  &                            \\
                  \text{c. } & \lim_{x\to a} [p(x) + q(x)] = \infty + \infty     &  & \text{false indeterminate}
                \end{align*}
          \item 4.
                \begin{align*}
                  \text{a. } & \lim_{x\to a} [f(x)]^{g(x)} = 0^0                                                                                                        \\
                  \text{b. } & \lim_{x\to a} [f(x)]^{p(x)} = 0^{\infty}                                                                 &  & \text{false indeterminate} \\
                  \text{c. } & \lim_{x\to a} [h(x)]^{p(x)} = 1^{\infty}                                                                                                 \\
                  \text{d. } & \lim_{x\to a} [p(x)]^{f(x)} = \infty^0                                                                                                   \\
                  \text{e. } & \lim_{x\to a} [p(x)]^{q(x)} = \infty^{\infty}                                                            &  & \text{false indeterminate} \\
                  \text{f. } & \lim_{x\to a} \sqrt[q(x)]{p(x)} = \sqrt[\infty]{\infty} = \infty^{\frac{1}{\infty}} = \infty^0 \text{ ?}
                \end{align*}
        \end{description}
  \item\textbf{Find the limit. Use L'Hospital's Rule if possible.}
        \begin{description}
          \setlength\itemsep{2em}
          \item 13. $\lim_{x\to\pi/4} \dfrac{\sin{x} - \cos{x}}{\tan{x} - 1}$
                \begin{align*}
                   & = \frac{\sin{\left(\frac{\pi}{4}\right)} - \cos{\left(\frac{\pi}{4}\right)}}{\tan{\left(\frac{\pi}{4}\right)} - 1} \\[1em]
                   & = \frac{\frac{\sqrt{2}}{2} - \frac{\sqrt{2}}{2}}{1 - 1}                                                            \\[1em]
                \end{align*}

                L'Hospital's Rule is applicable
                \begin{align*}
                   & \lim_{x\to\pi/4}\frac{\frac{dx}{dy}(\sin{x} - \cos{x})}{\frac{dx}{dy}(\tan{x} - 1)}                            \\[1em]
                   & = \lim_{x\to\pi/4}\frac{\cos{x}+\sin{x}}{\sec^2{x}}                                                            \\[1em]
                   & = \frac{\cos{\left(\frac{\pi}{4}\right)}+\sin{\left(\frac{\pi}{4}\right)}}{\sec^2{\left(\frac{\pi}{4}\right)}} \\[1em]
                   & = \frac{\frac{\sqrt{2}}{2}+\frac{\sqrt{2}}{2}}{\sqrt{2}^2}                                                     \\[1em]
                   & = \frac{\frac{2\sqrt{2}}{2}}{2}                                                                                \\[1em]
                   & = \boxed{\frac{\sqrt{2}}{2}}
                \end{align*}
          \item 17. $\lim_{x\to1} \dfrac{\sin{(x-1)}}{x^3 + x - 2}$
                \begin{align*}
                   & = \frac{\sin{(1-1)}}{1^3 + 1 - 2} \\
                   & = \frac{0}{0}
                \end{align*}

                L'Hospital's Rule is applicable
                \begin{align*}
                   & \lim_{x\to1}\frac{\frac{dx}{dy}(\sin{(x-1)})}{\frac{dx}{dy}(x^3 + x - 2)} \\[1em]
                   & = \lim_{x\to1}\frac{\cos{(x-1)}(1)}{(3x)^2 + 1}                           \\[1em]
                   & = \frac{\cos{(1-1)}}{3(1)^2 + 1}                                          \\[1em]
                   & = \frac{\cos{(0)}}{4}                                                     \\[1em]
                   & = \boxed{\frac{1}{4}}
                \end{align*}
          \item 21. $\lim_{x\to0^{+}} \dfrac{\ln{x}}{x}$
                \begin{align*}
                   & = \frac{\ln{0}}{0}  \\[1em]
                   & = \frac{-\infty}{1} \\[1em]
                   & = \boxed{-\infty}
                \end{align*}
          \item 25. $\lim_{x\to0} \dfrac{\sqrt{1 + 2x} - \sqrt{1 - 4x}}{x}$
                \begin{align*}
                   & = \frac{\sqrt{1 + 2(0)} - \sqrt{1 - 4(0)}}{0}
                \end{align*}

                L'Hospital's Rule is applicable
                \begin{align*}
                   & \lim_{x\to0}\frac{\frac{dx}{dy}(\sqrt{1 + 2x} - \sqrt{1 - 4x})}{\frac{dx}{dy}(x)}     \\[1em]
                   & = \lim_{x\to0}\frac{\frac{dx}{dy}((1 + 2x)^{1/2} - (1 - 4x)^{1/2})}{\frac{dx}{dy}(x)} \\[1em]
                   & = \lim_{x\to0}\frac{(1 + 2x)^{-1/2}-  (-2)(1 - 4x)^{-1/2}}{1}                         \\[1em]
                   & = \lim_{x\to0}\frac{\frac{1}{(1 + 2x)^{1/2}} +  \frac{2}{(1 - 4x)^{1/2}}}{1}          \\[1em]
                   & = \frac{\frac{1}{(1 + 2(0))^{1/2}} +  \frac{2}{(1 - 4(0))^{1/2}}}{1}                  \\[1em]
                   & = \frac{1 +  2}{1}                                                                    \\[1em]
                   & = \boxed{3}
                \end{align*}
          \item 29. $\lim_{x\to0} \dfrac{\tanh{x}}{tan{x}}$
                \begin{align*}
                  \frac{\tanh{0}}{\tan{0}} &  & \text{Using} \tanh{x}= \frac{e^x - e^{-x}}{e^x + e^{-x}} \\[1em]
                  \frac{\tanh{0}}{\tan{0}}                                                               \\[1em]
                  \frac{0}{0}
                \end{align*}

                L'Hospital's Rule is applicable
                \begin{align*}
                   & \frac{\frac{dx}{dy}(\tanh{x})}{\frac{dx}{dy}(\tan{x})}                                                    \\[1em]
                   & =    \frac{\sech^2{0}}{\sec^2{0}}                      &  & \text{Using} \sech{x}= \frac{1}{e^x - e^{-x}} \\[1em]
                   & =    \frac{1}{1}                                                                                          \\[1em]
                   & =    \boxed{1}
                \end{align*}
          \item 33. $\lim_{x\to0} \dfrac{x3^x}{3^x-1}$
                \begin{align*}
                   & =\frac{(0)3^0}{3^(0)-1} \\[1em]
                   & =\frac{0}{0}            \\
                \end{align*}

                L'Hospital's Rule is applicable
                \begin{align*}
                   & \lim_{x\to0}\frac{\frac{dx}{dy}(x3^x)}{\frac{dx}{dy}(3^x-1)}  \\[1em]
                   & = \lim_{x\to0}\frac{x(3^x)(\ln{3}) + (1)(3^x)}{(3^x)(\ln{3})} \\[1em]
                   & = \frac{0(3^0)(\ln{3}) + (1)(3^0)}{(3^0)(\ln{3})}             \\[1em]
                   & =  \frac{0 + 1}{\ln{3}}                                       \\[1em]
                   & =  \boxed{\frac{1}{\ln{3}}}
                \end{align*}
          \item 37. $\lim_{x\to0^{+}} \dfrac{\arctan({2x})}{\ln{x}}$
                \begin{align*}
                   & = \frac{\arctan({2(0)})}{\ln{0}} \\[1em]
                   & = \frac{0}{-\infty}
                \end{align*}

                L'Hospital's Rule is applicable
                \begin{align*}
                   & \lim_{x\to0^{+}}\frac{\frac{dx}{dy}(\arctan({2x}))}{\frac{dx}{dy}(\ln{x})}                                   \\[1em]
                   & = \lim_{x\to0^{+}}\frac{\frac{1}{1 + x^2}}{\frac{1}{x}}                                                      \\[1em]
                   & = \frac{\frac{1}{1 + (0)^2}}{\frac{1}{0}}                                                                    \\[1em]
                   & = \frac{\frac{1}{1 + 0}}{\infty}                                           &  & \infty\text{ because } 0^{+} \\[1em]
                   & =\boxed{0}
                \end{align*}
          \item 41. $\lim_{x\to0} \dfrac{\cos{x} - 1 + \frac{1}{2}x^2}{x^4}$
                \begin{align*}
                   & = \frac{\cos{0} - 1 + \frac{1}{2}0^2}{0^4} \\[1em]
                   & = \frac{1 - 1 + \frac{1}{2}}{0}            \\[1em]
                \end{align*}

                L'Hospital's Rule is applicable (substitution not shown)
                \begin{align*}
                   & \lim_{x\to0}\frac{-\sin{x} + x}{4x^3}
                \end{align*}

                L'Hospital's Rule is applicable (substitution not shown)
                \begin{align*}
                   & \lim_{x\to0}\frac{-\cos{x} + 1}{12x^2}
                \end{align*}

                L'Hospital's Rule is applicable (substitution not shown)
                \begin{align*}
                   & \lim_{x\to0}\frac{\sin{x}}{24x}
                \end{align*}

                L'Hospital's Rule is applicable
                \begin{align*}
                   & \lim_{x\to0}\frac{\cos{x}}{24} \\[1em]
                   & \frac{\cos{0}}{24}             \\[1em]
                   & =  \boxed{\frac{1}{24}}
                \end{align*}
          \item 45. $\lim_{x\to0}\sin{(5x)}\csc{(3x)}$
                \begin{align*}
                   & =  \sin{0}\csc{0}           \\[1em]
                   & = \frac{\sin{5x}}{\tan{3x}} \\[1em]
                   & = \frac{\sin{0}}{\tan{0}}   \\[1em]
                   & =  \frac{0}{0}              \\
                \end{align*}

                L'Hospital's Rule is applicable
                \begin{align*}
                   & = \lim_{x\to0}\frac{\frac{dx}{dy}(\sin{5x})}{\frac{dx}{dy}(\tan{3x})} \\[1em]
                   & = \lim_{x\to0}\frac{\cos{5x}(5)}{\sec^2{3x}(3)}                       \\[1em]
                   & \frac{\cos{0}(5)}{\sec^2{0}(3)}                                       \\[1em]
                   & = \frac{1(5)}{1(3)}                                                   \\[1em]
                   & = \boxed{\frac{5}{3}}
                \end{align*}
          \item 49. $\lim_{x\to1^+}\ln{x}\tan{(\pi x/2)}$
                \begin{align*}
                   & = \lim_{x\to1^+}\frac{\ln{(\pi x/2)}\sin{(\pi x/2)}}{\cos{(\pi x/2)}} \\[1em]
                   & =  \frac{\ln{(\pi /2)}\sin{(\pi /2)}}{\cos{(\pi /2)}}                 \\[1em]
                   & =  \frac{\ln{(\pi /2)}}{0}
                \end{align*}

                L'Hospital's Rule is applicable.
                \begin{align*}
                   & = \lim_{x\to1^+}\frac{\frac{dx}{dy}(\ln{(\pi x/2)}\sin{(\pi x/2)})}{\frac{dx}{dy}(\cos{(\pi x/2)})}                                   \\[1em]
                   & = \lim_{x\to1^+}\frac{\ln{(\pi x/2)}\cos{(\pi x/2)}(\pi /2)  +  \frac{1}{(\pi x/2)}(\pi /2)\sin{(\pi x/2)}}{-\sin{(\pi x/2)}(\pi /2)} \\[1em]
                   & = \frac{\ln{(\pi /2)}\cos{(\pi /2)}(\pi /2)  +  \frac{1}{(\pi /2)}(\pi /2)\sin{(\pi /2)}}{-\sin{(\pi /2)}(\pi /2)}                    \\[1em]
                   & = \frac{\ln{(\pi /2)}(0)(\pi /2)  +  1(1)}{-1(\pi /2)}                                                                                \\[1em]
                   & = -\frac{1}{\pi /2}                                                                                                                   \\[1em]
                   & = \boxed{-\frac{2}{\pi }}
                \end{align*}
          \item 53. $\lim_{x\to0^+}\dfrac{1}{x} - \dfrac{1}{e^x-1}$
                \begin{align*}
                   & =  \lim_{x\to0^+}\frac{1}{x} - \frac{1}{e^x-1}                                             \\[1em]
                   & =  \lim_{x\to0^+}\frac{1}{x} \cdot \frac{e^x-1}{e^x-1} - \frac{1}{e^x-1} \cdot \frac{x}{x} \\[1em]
                   & =  \lim_{x\to0^+}\frac{(e^x-1) - 1}{x(e^x-1)}                                              \\[1em]
                   & =  \lim_{x\to0^+}\frac{e^x-2}{xe^x-x}                                                      \\[1em]
                   & =  \frac{-1}{0}
                \end{align*}

                L'Hospital's Rule is applicable.
                \begin{align*}
                   & \lim_{x\to0^+}\frac{\frac{dx}{dy}(e^x-2)}{\frac{dx}{dy}(xe^x-x)} \\[1em]
                   & =  \lim_{x\to0^+}\frac{e^x}{xe^x + 1e^x -1}                      \\[1em]
                   & =  \frac{e^0}{0(e^0) + 1(e^0) -1}                                \\[1em]
                   & =  \frac{1}{1 - 1}                                               \\[1em]
                \end{align*}

                L'Hospital's Rule is applicable.
                \begin{align*}
                   & \lim_{x\to0^+}\frac{\frac{dx}{dy}(e^x)}{\frac{dx}{dy}(xe^x + e^x -1)} \\[1em]
                   & = \lim_{x\to0^+}\frac{e^x}{xe^x + 1e^x + e^x}                         \\[1em]
                   & = \frac{e^0}{0e^0 + 1e^0 + e^0}                                       \\[1em]
                   & = \frac{1}{0 + 1 + 1}                                                 \\[1em]
                   & = \boxed{\frac{1}{2}}
                \end{align*}
          \item 57. $\lim_{x\to0^+}x^{\sqrt{x}}$
                \begin{align*}
                   & = \lim_{x\to0^+}x^{\sqrt{x}}                                \\[1em]
                   & = \lim_{x\to0^+}e^{\sqrt{x}\ln{x}}                          \\[1em]
                   & = \lim_{x\to0^+}e^{\sqrt{0}\ln{0}}                          \\[1em]
                   & = \boxed{1}                        &  & \text{because } 0^+
                \end{align*}
          \item 61. $\lim_{x\to1^+}x^{1/(1-x)}$
                \begin{align*}
                   & =  \lim_{x\to1^+} e^{\ln(x^{1/(1-x)})} \\[1em]
                   & =   e^{\lim_{x\to1^+}\ln(x^{1/(1-x)})}
                \end{align*}

                Isolate the exponent.
                \begin{align*}
                   & \lim_{x\to1^+}\ln{(x^{1/(1-x)})}      \\[1em]
                   & =   \lim_{x\to1^+}\frac{1}{1-x}\ln{x} \\[1em]
                   & =   \frac{1}{1-1}\ln{1}               \\[1em]
                   & =   \frac{0}{0}
                \end{align*}

                L'Hospital's Rule is applicable.
                \begin{align*}
                   & \lim_{x\to1^+}\frac{\frac{dx}{dy}(\ln{x})}{\frac{dx}{dy}(1-x)} \\[1em]
                   & =   \lim_{x\to1^+}\frac{\frac{1}{x}}{-1}                       \\[1em]
                   & =   \lim_{x\to1^+} - \frac{1}{x}                               \\[1em]
                   & =    - \frac{1}{1}                                             \\[1em]
                   & =    - 1                                                       \\[1em]
                \end{align*}

                Substitute exponent into modified original function.
                \begin{align*}
                   & e^{(-1)}               \\[1em]
                   & =  \boxed{\frac{1}{e}}
                \end{align*}
          \item 65. $\lim_{x\to0^+}(4x + 1)^{\cot{x}}$
                \begin{align*}
                   & = \lim_{x\to0^+}e^{\ln{((4x + 1)^{\cot{x}})}}   \\[1em]
                   & = e^{\lim_{x\to0^+}(\ln{((4x + 1)^{\cot{x}})})} \\[1em]
                \end{align*}

                Isolate the exponent.
                \begin{align*}
                   & \lim_{x\to0^+}\ln{((4x + 1)^{\cot{x}})}              \\[1em]
                   & = \lim_{x\to0^+}\cot{x}\ln{(4x + 1)}                 \\[1em]
                   & = \lim_{x\to0^+}\frac{\ln{(4x + 1)\cos{x}}}{\sin{x}} \\[1em]
                   & = \frac{\ln{(4(0) + 1)\cos{0}}}{\sin{0}}             \\[1em]
                   & = \frac{0 \cdot 1}{0}
                \end{align*}

                L'Hospital's Rule is applicable.
                \begin{align*}
                   & = \lim_{x\to0^+}\frac{\frac{dx}{dy}(\ln{(4x + 1)\cos{x}})}{\frac{dx}{dy}(\sin{x})}       \\[1em]
                   & = \lim_{x\to0^+}\frac{\ln{(4x + 1)}(-\sin{x}) + \frac{1}{(4x + 1)}(4)(\cos{x})}{\cos{x}} \\[1em]
                   & = \frac{\ln{(4(0) + 1)}(-\sin{0}) + \frac{1}{(4(0) + 1)}(4)(\cos{0})}{\cos{0}}           \\[1em]
                   & = \frac{(0)(0) + \frac{1}{1}(4)(1)}{1}                                                   \\[1em]
                   & = 4                                                                                      \\[1em]
                \end{align*}

                Substitute exponent into modified original function.
                \begin{align*}
                   & \boxed{e^{4}}
                \end{align*}
        \end{description}
\end{description}
\end{document}
